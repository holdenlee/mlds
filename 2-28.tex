\section{2019-2-28 Nonlinear filtering}

Outline:
\begin{itemize}
\item
Zakai and Kushner equiations
\item
Girsanov theory
\item
Derivation
\item
Particle filtering
\item
Deep learning to solve Zakai
\end{itemize}

\subsection{Zakai and Kushner equation}

The hidden state evolves as 
\begin{align}
dX_t &= f_t(X_t)\,dt +\si \, dW_t.
\end{align}
The observation is 
\begin{align}
dY_t &= h_t(X_t)\,dt + \eta dB_t.
\end{align}
Here, $B_t$, $W_t$ are independent Brownian motion. The goal is to calculate the posterior
\begin{align}
p_t(x) &= p(X_t=x|Y_s=y_s,0\le s\le t).
\end{align}
The first term is from Fokker-Planck, the second term is variance reduction from the observations.  
%The amount variance is reduced is proportional to
%observe some linear combination, cut that direction.
%effect of second term: 
%observation: variance reduction scheme
%density within in subspace, small variance outside?
For the Kalman filter, covariance matrix is subtracted by a positive definite matrix. The same thing is going on here.
%cov of this p_t is smaller than cov of FK p_t

The Kushner equation solves for the posterior density.
\begin{align}
dp_t(x) &=\cal L^* [p_t]\,dt + p_t(x)[h_t(x)-\E h_t]^\top  \eta^{-\top}\eta^{-1}[dy_t-\E h_t].
\end{align}
Here %$\cal L^*$ is the adjoint of
\begin{align}
\cal L &= f \pdd x + \rc 2(\si\si^\top) \pdt{}{x}\\
\cal L^*[p_t] &= -\nb \cdot [f_t\cdot p_t]+\rc 2 (\si\si^\top) \cdot \pl^2 p_t\\
dg(x_t)&=\cal L[g]\,dt\\
\E_t h_t &=\int h_t(x)p_t(x)\dx
\end{align}
We have to update both; this is numerically not easy. We introduce an unnormalized verision that is easier.

Zakai gives an unnormalized density,
\begin{align}
d\wt p_t(x) &=\cal L^* [\wt p_t] \,dt + \wt p_t(x) h_t^\top(x)\eta^{-\top}\eta^{-1}\,dy_t.
\end{align}
To say it is an unnormalized density means
\begin{align}
\wt p_t(x) &= \al(t)p_t(x)\\
\int \wt p_t(x) &\ne 1\text{ in general.}
\end{align}
\begin{rem}
Assume $Y_t/\eta$ is Brownian motion; let $Q$ be its law. Let $\dd QP=M_T^{-1}$. Define 
\begin{align}
\ph_t[g]:&=\E_Q[g(X_t)M_t|\cal F_t^Y]\\
\psi_t[g]:&= \E_P[g(X_t)|\cal F_t^Y] = \fc{\ph_t[g]}{\ph_t[1]}.
\end{align}
$\al(t):=\ph_t[1]$. 
%Note $|\E_Q[M_t|\cal F_T^Y]|=1$. So it won't be too large or small, more like a fluctuation.
\end{rem}

The forward form of the Kushner equation is
\begin{align}
\psi_t[g]&=\E[g(X_t)|\cal F_t^Y]\\
d\psi_t[g] &=\psi_t[\cal L g]\,dt 
+ \fc{\psi_t[gh]-\psi_t[g]\psi_t[h]}{\eta^2} (dY_t-\psi_t[h]\,dt)
\end{align}
To show the variance is smaller than in the Fokker-Planck equation we can take $g(x)=x^2$.

\begin{rem}
If there is no noise, and $f$ is unknown, this is a ML problem (cf. RNN). For Kalman filtering if we know the model, it's not ML; if we don't know the model, we have to do system identification, this is ML. 

Suppose $f_t,h_t=0$, and we only have noise. It's just Brownian motion. If we have observations up to time $t$, that means the system picked a particular choice of Brownian motion. We don't know the particular path. If we have observations, we know something about the path. The variance is smaller, we solve a filtering problem. 

Filtering by itself is variance reduction (through observations). Estimation is machine learning.
\end{rem}

%instead of deriv, spend time understanding what obs means
\subsection{Girsanov theory}
We use a change of measure to make the integration easier.
Let's consider a simple example first.
\begin{ex}
Let $(\Om,\cal F,P)$ be a probability space and $\xi\sim N(0,1)$ under $P$. For $a\in \R$, then $\xi'=\xi+a\sim N(a,1)$ under $P$. We want to find $Q$ such that $\xi'\sim N(0,1)$ under $Q$. 

For any $A\in B(\R)$, 
\begin{align}
P(\xi'\in A) &=\int_A \rc{\sqrt{2\pi}} e^{-\rc 2(x-a)^2}\dx\\
Q(\xi'\in A) &=\int_A \rc{\sqrt{2\pi}}e^{-\rc 2x^2}\dx\\
\dd QP :&= e^{-\rc 2 \xi^{\prime2} +\rc 2(\xi^{\prime2}-a)^2} = e^{-a\xi' + \rc 2a^2} = e^{-a\xi-\rc 2a^2}.
\end{align}
Then
\begin{align}
\E_Q[f(\xi')]&=\E_Q [f(\xi+a)] = \E_p[f(\xi+a)e^{-a\xi -\rc 2a^2}]=\E_p [f(\xi)].
\end{align}
\end{ex}
We can also do the change-of-variable for a random process, using Girsanov. We will use it to remove drift $h(X_t)$,  and make $Y_t$ Brownian motion.
\begin{thm}[Girsanov] Let $(\Om, \cal F, \{\cal F_i\}, P)$ be a probability space with filtration, $B$ be $n$-dimensional Brownian motion, $X\in \cal H_{\text{loc}}^2[0,T]$. Define 
\begin{align}
\dd QP &= M_T:= \exp\pa{-\int_0^T X_s\,dB_s - \rc 2 \int_0^T \ve{X_s}^2\,ds}\\
B_t'&=B_t+\int_0^t X_s\,ds.
\end{align}
%stoch process version of change of measure
%match up with -a\xi and -\rc 2 a^2
If $\E_PM_T=1$, then $(B_t')_{t\le T}$ is Brownian motion under $Q$. Also, $M_t$ is a $P$-martingale. 
\end{thm}
\begin{rem}[Novikov]
If $\E\exp\pa{\rc 2\int X_s^2\,ds}<\iy$, then $\E M_T<1$.
\end{rem}

%hidden markov process
%viterbi wrong alg. baum-welch relies on viterbi?

\subsection{Derivation}

%change measure so $Y$ is BM; then independent, so easier.
Omitted.

\subsection{Particle filtering}

Particle filtering is Monte Carlo sampling with modifications based on observations.

We know $p(X_0)$, $p(X_t|X_{t-1})$, $p(Y_t|X_t)$, and have observations $Y_{1:T}=y_{1:T}$.
The goal is to sample $\xi_t\sim p(X_t|Y_1,\ldots, Y_t)$. %=:\pi_t$. 

The algorithm is as follows.
\begin{alg}[Particle filtering]
\begin{itemize}
\item
Let $\pi_0=p(X_0)$. 
\item
For $t=1,\ldots, T$, do:
\begin{itemize}
\item
(Resampling)
Sample $\xi_{t-1}^{(i)} \sim \pi_{t-1}$, iid, for $i=1,\ldots, N$. 
\item
Sample $\xi_{t-\rc 2}^{(i)}\sim p(X_t|X_{t-1}=\xi_{t-1}^{(i)})$ (from the model). %predict prior
\item
Compute
\begin{align}
w_t^{(i)} &= \fc{p(Y_t=y_t|x_t=\xi_{t-\rc 2}^{(i)})}{\sumo jN p(Y_t=y_t|X_t=\xi_{t-\rc 2}^{(j)})}
\end{align}
\item
Let $\pi_t = \sumo iN w_t^{(i)} \de_{\xi_{t-\rc 2}}^{(i)}$.
\end{itemize}•
\end{itemize}
\end{alg}
When sampling $\pi_{t-1}$, not all points will be sampled and some will be sampled more than once. 
%some not go on, some will be sampled >once

The rate is $\rc{\sqrt N}$, but the constant depends on the dimension and increases exponentially. If we make some assumptions on the $X$, e.g. sparsity of interactions (and particles far away are not correlated), %then we can sample based on the dimensions with higher correlation
we can do better.

(Ramon van Handel's paper: can we overcome curse of dimensionality?Conclusion, not so clear.)

%define effective size. long distance not correlated
\subsection{Deep learning for solving Zakai}

Numerical method to solve Zakai or Kushner equations directly.
%use deep NN to approx first order derivative

Zakai: $d\wt p_t(x) = \cal L^*[\wt p_t]\,dt + \wt p_t(x) \fc{h_t(x)}{\eta^2}\,dy_t$.

The method can help us calculate $\int g(x) p_t(x)\dx$. %not $p_t(x^*)$ because that would require taking $g(x)=\de_{x^*}$.
%econ: state variables.
%math well defined. numerically?

%Given $U_T$
Given $U_T$, generate a backwards path
\begin{align}
U_t&=U_{t+1}-f_t(U_{t+1})\De t + \si \De W_t\\
V_t&=p_t(U_t)\\
V_{t+1}-V_t&=p_{t+1}(U_{t+1})
-p_t(U_{t+1}) + p_t(U_{t+1}) - p_t(U_t)\\
%$2p \De u_t + \rc 2 \pl^2 p(\De u_t)^2$
&=[-\nb \cdot (f_t(U_{t+1})p_t(U_{t+1})) + \rc2 \si\si^\top \pl^2p_t(U_{t+1})]\,dt\\
&\quad +p_t(U_{t+1}) \fc{h_t(U_{t+1})}{\eta^2}\De Y_t + o(\De t)\\
&=-V_{t+1}\nb \cdot f_t(U_{t+1}) - \pl p_t(U_{t+1})\cdot \si\De t
+V_{t+1}h_t^\top (U_{t+1}) \eta^{-\top} \eta^{-1} [\De Y_t - h_t(U_{t+1})\De t] + o(\De t)
\end{align}
%Updates of $V$ only depend on 
The only unknown term is $-\pl p_t(U_{t+1})$.
%Let it be $\ph_t(U_{t+1})$. 
We can calculate $V_t=P_t(U_t)$ %\psi(U_T)
 from $T$ to 0. We know $V_0=p_0(U_0)$. We minimize the loss function,
 \begin{align}
\min_{\ph,\psi} \sum_{\text{samples }U_T,\De W_t} |V_0-p_0(U_0)|^2.
\end{align}
%Dropping the constraints is better. We cannot ensure $\ph_t(U_{t+1})=\pl p_t(U_{t+1})$. 
We tested it on the Kalman filter in 2 dimensions.
%replace $\psi$ by $V_T$, don't have $U_T$.
%not good enough
%preferable to solve Zakai instead of Kushner.
%calc $\pl p_t(U_{t+1})$ numerically?