\section{2019-3-21 Strategies for filtering turbulent systems: Particle filtering (Di Qi)}

\subsection{Strategies for filtering turbulent systems}

Particle filtering suffers from the curse of dimensionality.
%adding proper amount of noise.
%Xingtong

Fundamental challenges for real-time filtering of turbulent signals: model errors, limited ensemble size (50--100 for state space dimension $10^4$ to $10^8$), sparse noisy spatio-temporal observations for partial set of variables.

Filtering is the process of obtaining the best statistical estimate of a natural system from partial observations of the true signal. 
\begin{align}p_{m+1,+}(u) &= \fc{p_{m+1}(v_{m+1}|u)p_{m+1,-}(u)}{\int p_{m+1}(v_{m+1}|u)p_{m+1,-}(u)\,du}.\end{align}
(Normalization: divide by sum.)

Step 1: forecast, step 2: correction. Observation reduces variance. Filtering is also called data assimiliation.
%combine model fore

Example application: predicting path of hurricane. Large forecast changes for Hurricane Katrina in a few hours.

Theoretical and computational issues:
\begin{itemize}
\item
Handling nonlinearity: convergence requires ensemble size growing exponentially with respec tto ensemble spread relative to observation errors  (Bengtsson, Bickel, Li)
\item
Large systems
\item
Computational bottleneck: propagating covariance matrix of size $N\times N$.
\item
Strategies: Ensemble Kalman filters (ETKF, EAKF). Each requires computing SVD.
\item
Catastrophic filter divergence (go to machine $\iy$) when observations are sparse, even when true signal is dissipative system with ``absorbing ball property."
\end{itemize}
Kalman filter optimal for linear system. We need to work with a high-dimensional covariance matrix.

Particle filter: with prior, get posterior weights and distribution
\begin{align}
p_-(u)&=\sum_j p_{j,-}\de(u-u_j)\\
p_{j,+}&\propto p(v|u_j) p_{j,-}\\
p_+(u) &=\sum_j p_{j,+}\de(u-u_j).
\end{align}
Capture non-gaussian statistics.

Problem: After running for several steps, weight concentrates on one particle. (Because exponential decay of probability from MLE.) This is why correction methods are proposed. EnKF maintains ensemble, but only uses it to get the mean and covariance; make a gaussian fit to the nongaussian distribution.
%reduce cost
%can apply to nonlinear, but only extract gaussian feature

Particle filter vs. Kalman filter:
\begin{itemize}
\item
PF based on MC approaches with various resampling strategies provides better estimates of low-dimensional systems than KF in presence of strong nonlinearity and highly non-gaussian distributions.
\item
Less feasible for high-dimensional systems. %with large ensemble sizes.
%dim up, correlation goes down.
%block particle filtering
%cross correlations large in error
Computational constraints, particle collapsing.
\end{itemize}
Ideas for filtering high-dimensional turbulent systems: Blended method
\begin{itemize}
\item
Decompose into 2 subspaces $u=(u_1,u_2)$ changing adaptively in time
\item
Subspace $u_1$ with dimensionality $N_1$ is low-dimensional enough, capturing non-gaussian structures; accurate nonlinear filters (PF) can be used.
\item
Orthogonal subspace $u_2$ where gaussian statistics assumed: cheap Gaussian filters (KF) can be used.
\end{itemize}
Problems:
\begin{itemize}
\item
forecast: how to decompose space properly (time-dependently)
\item
analysis: observations mix the state variables in these two separate subspaces with different representations.
\end{itemize}•
\begin{align}
p_-(u_1,u_2)&= p_-(u_1)p_-^T(u_2|u_1)\\
&=\sumo jQ p_{j,-}\de(u_1-u_{1,j}^-)\cal N(u_{2,j}^-,R_{2,j}^-)\\
\ol u_{2,j}^- &=\int u_2 p^T(u_2,u_{1,j}^-)\,du_2\\
R_{2,j}^-&=\cdots
\end{align}
Compute that the posterior distribution also has this form. Analytic pdf given by Bayes.

Orthogonal subspace statistics with Gaussian mixture representation.
For $u_1$, adaptively evolving basis to keep track of most energetic directions of phase space.

%63
Lorenz 96 system: mimic large-scale behavior of mid-latitude atmosphere around circle of constant latitude. (First test problem for climate filtering problems.) 
$\dd{u_i}t = u_{i-1}(u_{i+1}-u_{i-2})-u_i+F_i$. Quadratic part conserves energy, $B(u,u)\cdot u=0$. Simplest example of complex turbulent dynamical system with properties in realistic systems.
%plot: grid points vs. time

Increasing $F$ breaks up the wave structure, and makes it more gaussian.

True signal generated by running model once to time $T=250$. Initial data: perturb initial truth state with Gaussian noise with climatological variance $R_\iy$.  Ensemble size $K=10000$, DO subspace $S=5$, test 2 different observation noise amplitude $r_0$ and observation time $\De t$. (Ex. sparse infrequent high quality observations in the ocean, $r_0=0.01$, $\De t=0.25$: often leads to catastrophic filter divergence.)
%sparse infrequent high quality observations, $F=8$, $r_0=0.01$, $\De t=0.25$, $p=4$.

To decompose space, find Fourier transform of 14 grid points. $u(t) = \sumo k{N_t} Z_k(t)\vec v_k(t)$, $\dd{\vec v_k}t$.

Two-layer quasigeostrophic equation.

%never diverge, good results
%climate filtering: First alw

%cubic sensor
What's a good realistic test problem?
%develop practical theory
%cannot go directly to practical models
%everything complicated
%code in FORTRAN

\subsection{Statistical reduced models and rigorous analysis for uncertainty quantification of turbulent geophysical flows}
%Andrew Majda

In $\R^N$
\begin{align}
\dd ut &= F(u,t) + \si(u,t)\dot W(t).
\end{align}
Sources of uncertainty: internal instabilities, initial and boundary conditions, model approx, limited observations and data.

Challenges: non-gaussian, large dim phase space, non-stationary dynamics, wide range of scales.

Obtain accurate stat estimates such as change in mean and variance for key statistical quantities in nonlinear response to changes in external forcing or uncertain initial data.

Imperfect model $u_M$, $F_M$, $\si_M$. $u_M\in \R^M$, want $M\ll N$.

Examples: 
\begin{itemize}
\item
geophysical fluid flows, plasma flow,
%Tokamak - geostrophic equations
high Reynolds numbers.
\item
systems biology, structural systems, US power network.
\end{itemize}

General framework: with energy preserving quadratic part
\begin{align}
\dd ut &=\cL[u(t;\om);\om] = (L+D)u+B(u,u) + F(t)+\si(t)\dot W(t;\om),
\end{align}•
$L^*=-L$, $D\le 0$, $u\cdot B(u,u)=0$.

Model selection (ergodic theory, statistical measures in equilibrium), model calibration (empirical information theory, linear response theory, total statistical energy equations), model prediction (numerical stability).

Damping and forcing tems. Can find exact invariant measure
\begin{align}
\dd ut &= \cL[u] = B(u,u)+Lu-d\La u + \La^{\rc 2}\si \dot W(t;\om).
\end{align}
Decompose state variable into mean state and fluctuation around mean, $u(t) = \ol u(t) + \sum Z_i(t)v_i$, $R_{ij}=\an{Z_iZ_j^*}$.
The system will never be closed, because there are 2nd order terms.
\begin{align}
\dd{\ol u}t &=(L+D)\ol u + B(\ol u,\ol u) + R_{ij}\an{v_i,v_j}+F(t)\\
\dd Rt &=L_vR+RL_v^*+Q_F+Q_\si.
\end{align}
Nonlinear flux. Energy is still conserved.

Example: one-layer barotropic flow with topography (simplest model in geophysical systems). Kinetic energy and large scale enstrophy conserved.

Two-layer, more complex.
%3rd order  moments, stable unstable

\subsubsection{Reduced-order statistical model with consistency and sensitivity}

Huge dynamical system in background. Smaller dynamical system with similar behavior for what you're interested in.

New systematic approach for nonlinear flux $Q_F^M$ combining detailed model energy mechanism and control over model sensitivity.

\begin{itemize}
\item
Model fidelity: guarantees reduced model converence to same final unperturbed statistical equilibrium.
Statistical energy principle offers a closed equation.
\item
Model sensitivity to external perturbations requires imperfect model to correctly reflect true system's memory.

Linear response operator characterizes model sensitivity regardless of specific perturbations.
For any function of state variable, the  expectation can be expanded as unperturbed equilibrium measure, plus linear response.
$\E^\de A(u) = E_{eq}(A)+\de E_A' + O(\de^2)$

Use relative entropy as distance between two probability densities.
\end{itemize}

Flow in low-latitude regimes with zonal jets. How do they respond to external forcing?
%1 dominant mode

Reduce size 256x256x2 to 10x10x2.

%reducing tracer intermittency
%passive tracer advected by flow
%interested in exponential tails, approximate

\subsubsection{Quantifying statistical reliability}

Nonlinear saturation of instabilities estimates the max growth in one unstable flow situation. %Statisticl stability: saturation of prob dist of sensemble of trajectories

%id fixed point.
Linear dependent solution is most probable state from maximm entropy principle.

Canonical statistical theory predicts invariant Gibbs measure for truncated barotropic equation.

%stat mean state and disturbance around mean.
Idea: use conserved total statistical energy in fluctuations.

%total stat energy decomp into pos-def and neg-def.
Slaving principle: perturbed mean and variance in all high wavenumber modes are ``slaved" by low wavenumber large-scale perturbations for all time. 

Usually many more small-scale modes.

%Geometric ergodicity under dissipation, inhomogeneous deterministic forcing and minimal stochastic forcing.

Instability: typically treat as initial value problem. Here, treated as statistical problem.

