\section{2019-3-7 System identification (Linear gaussian state space model)}

The model is 
\begin{align}
y_t&=Cx_t+v_t\\
x_{t+1}&=Ax_t + w_t
\end{align}
where $w_t\sim N(0,Q)$ and $v_t\sim N(0,R)$. 

In economics, all macroscopic discrete time models can be reduced to this formulation. 

%Identifiable: $P(A,C,Q,R|\{y_t\}_{t=1}^T)$.
The solution is by nature %non-identifiable: We
non-unique, we can replace $C$ with $C\Om^{-1}$ and $A$ with $\Om A\Om^{-1}$. This is non-essential.

We want to estimate $P(A,C,Q,R|\{y_t\}_{t=1}^T)$.  Let $\te=(A,C,Q,R)$. By Bayes,
%don't get conditional directly
\begin{align}
P(A,C,Q,R|\{y_t\}_{t=1}^T)
&\propto 
P(\{y_t\}_{t=1}^T|\te)P(\te)
\end{align}
We get this with the EM algorithm.
\begin{enumerate}
\item
Initialize $\te$. Use Kalman filtering/smoother to get $P(x_{1:T}|y_{1:T,\te})$  and $P(y_{1:T}|\te) = \prodo t{T-1} P(y_{t+1}|y_t,\te)P(y_1)$.
%em - conditional mean $\E[x_t|y_{1:T}]$.
Let $x_t$ be the conditional mean, $x_t=\E[x_t|y_{1:T}]$. 
\item
From $P(\te|x_{1:T}, y_{1:T})$ obtain $\wh\te_{MLE}$.
\end{enumerate}
The second (Bayesian) method is to do MCMC. Use the distributions instead: instead of using $\E[x_t|y_{1:T}]$, but take a lot of samples; likewise, instead of getting $\wh \te_{MLE}$, take a lot of samples. You need a prior; people use a conjugate prior to make things easier. 
%We have
%\begin{align}
%P(\te|x_{1:T}, y_{1:T})
%\end{align}
Explicitly,
\begin{align}
\ell(A,C,Q,R|x_{1:T},y_{1:T})
&=\fc T2 \log |Q^{-1}| - 
\rc 2 \Tr\ba{
Q^{-1}\pa{\sumz t{T-1}\cdots}
}
\end{align}
(The variance is also estimated so $P(\te,x_{1:T},y_{1:T})$ is gaussian.)

The third method is principal component estimation, when $N\gg K$. People put a normalization condition called principal component normalization.
Say $x_t\in \R^{F\times 1}$ and $y_t\in \R^{N\times 1}$. Then PC normalization has $k^2$ degrees of freedom, $\rc k C^\top C = I_k$ and $\E[X_t^\top X_t]$ is diagonal.
%estimate 

Run PCA on $y_t$, get first $K$ components. 
From the principal components you have $C$, and then you estimate $A$ separately.
%full col rank
If the second equation is nonlinear, can you still do it?
%There's an equivalent condition for when you can use PCA: 
You can use PCA when $\Var(y_t)=I$ (steady state) %stedy-state 
and $\Var(y_t)=C^\top  C + R$.
%minimize function.

%EM: conditional mean.
%nonlinear: Zakai
%In principle this also works for the non-linear case.
The formulation of EM works in the nonlinear case (if you have functions for the dynamics depending on $\te$).

%Wishart prior
%uncertainty important in economics

%If nonstationary, convergence?
%when you generate A,C,Q,R, know $y_t$ stationary.
%HW: prove convergence or not convergence.
